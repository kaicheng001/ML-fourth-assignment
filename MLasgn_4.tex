%! TEX program = pdflatex

\documentclass[oneside,solution]{seu-ml-assign}

\title{Assignment}
\author{Xie xing}
\studentID{58122204}
\instructor{Xu-Ying Liu}
\date{\today}
\duedate{May 17, 2024}
\assignno{3}
\semester{SEU --- 2024 Spring}
\mainproblem{Model Selection and Evaluation \& Neural Networks}

\begin{document}

\maketitle

% \startsolution[print]
%第一题SVM
\problem{Support Vector Machine }

%第一题第一问
\subsection{}


%第一题第一问(a)
\subsubsection{(a)}
Generalized Lagrangian function:
\begin{equation}\begin{aligned} & L(\omega,b,\epsilon,\alpha,\mu)=\frac12||\omega||^2+C\sum_{i=1}^m\epsilon_i,+\sum_{i=1}^m
               \alpha_i(1-\epsilon_i-y_i(\omega^Tx_i+b))-\sum_{i=1}^m\mu_i\epsilon_i\end{aligned}
\end{equation}
The dual problem of the original function is
\begin{equation}\max_{\alpha\geq0,\mu\geq0}\min_{\omega,b,\epsilon}L(\omega,b,\epsilon,\alpha,\mu)\end{equation}

Find the partial derivative :
\begin{equation}
  \nabla_{w}L(w,b,\epsilon,\alpha,\mu)=w-\sum_{i=1}^{N}\alpha_{i}y_{i}x_{i}=0 
\end{equation}

\begin{equation}
  \nabla_{b}L(w,b,\epsilon,\alpha,\mu)=-\sum_{i=1}^{N}\alpha_{i}y_{i}=0
\end{equation}

\begin{equation}
  \nabla_{\epsilon_{i}}L(w,b,\epsilon,\alpha,\mu)=C-\alpha_{i}-\mu_{i}=0
\end{equation}

Solutions have to:
\begin{equation}\left.\left\{\begin{aligned}w&=\sum_{i=1}^N\alpha_iy_ix_i\\
  \sum_{i=1}^N\alpha_iy_i&=0\\
  C-\alpha_i-\mu_i&=0\end{aligned}\right.\right.
\end{equation}
Bringing in $L(w,b,\epsilon_i,\alpha,\mu)$ gets:

\begin{equation}L(w,b,\epsilon_i,\alpha_i,\mu_i)=-\frac12\sum_{i=1}^N
  \sum_{j=1}^N\alpha_i\alpha_ju_iy_j\left(x_i\cdot x_j\right)+\sum_{i=1}^N\alpha_i
\end{equation}

Next, we originally found the maximum value of the above equation. 
If we add a negative sign to the entire equation, we can transform it into finding its minimum value, that is,
\begin{equation}\min_\alpha\frac12\sum_{i=1}^N\sum_{j=1}^N\alpha_i\alpha_ju_iy_j\left(x_i\right.\cdot x_j)-\sum_{i=1}^N\alpha \end{equation}

After getting the objective function, sort out the constraints.
First of all, there is a partial derivative solution $\sum_{i=1}^N\alpha_iy_i=0$,;\\
Secondly, the Lagrange multiplier is greater than or equal to 0, that is $\alpha, \mu >=0$, 
when seeking partial derivatives, we get $C-\alpha_i-\mu_i=0$;\\
Finally, comprehensively we get $0\leq\alpha_i\leq C$.

So the dual problem is:



%第一题第一问(b)
\subsubsection{(b)}


%第一题第二问
\subsection{}
%第一题第二问(a)
\subsubsection{(a)}

The corresponding mapping function is:
\begin{equation}\phi(x)=(x_1^2,\sqrt{2}x_1x_2,x_2^2)\end{equation}


%第一题第二问(b)
\subsubsection{(b)}

%第一题第二问(c)
\subsubsection{(c)}



\vspace{2mm}
\end{document}
